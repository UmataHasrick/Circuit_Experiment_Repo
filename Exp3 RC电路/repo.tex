\documentclass[lang=cn,11pt]{elegantpaper}

\usepackage{mathrsfs}

\usepackage{slashed}

\title{RC电路的频率特性}
\author{PB18020556 戴佳乐 PB19000132 苗立扬}



% 不需要版本信息,直接注释即可
%\version{0.07}
% 不需要时间信息的话,需要把 \today 删除。
\date{}


% 如果想修改参考文献样式,请把这行注释掉
%\usepackage[authoryear]{gbt7714}  % 国标

\begin{document}

\maketitle

\section{实验目的}
1. 熟悉正弦稳态分析中的相量的基本概念。

2. 正确使用双踪示波器测量正弦信号的峰一峰值 Up-p,频率 $\mathrm{f}(\mathrm{T})$ 和相位差 $\varphi$,观察李沙育图形;学会 使用晶体管毫伏表测量正弦信号有效值。

3. 用 RC、RL 设计输出滞后(超前)输入的简单电路,并作实际测量。

\section{实验仪器}
1. DF1641D 型或 EE1641D 型函数发生器 1 台

2. 双踪示波器 1 台

3. 晶体管亳幅表 DF2173B 1 台

4. 可变电容箱 1 个

5. 可变电阻箱 1 个

6. 可变电感箱 1 个

\section{实验原理}
1. 正弦交流电作用于任一线性定常电路, 产生的响应仍是同频率的正弦量,因此,正弦量可以用相量来表示。设一正弦电流:
$$
\begin{aligned}
&i(t)=\sqrt{2} I \operatorname{COS}\left(\omega t+\varphi_{i}\right)=R_{e}\left[\sqrt{2} I e^{j \varphi_{i}+j \omega t}\right] \\
&R_{e}\left[\sqrt{2} \dot{I} e^{j \omega t}\right] \leftrightarrow \dot{I}=I e^{j \varphi_{i}}
\end{aligned}
$$

2. 用相量表示了正弦量,正弦交流稳态响应的计算可方便地运用相量进行复数运算,在直流电路中的基本定律、定理和计算方法完全适用于相量计算。

3. 输出电压滞后输入电压的 RC 电路,如图 1 所示。

输出电压 $\dot{U}_{0}=\frac{1 / j \omega C}{R+1 / j \omega C} \dot{U}_{i}=\frac{\dot{U}_{i}}{j \omega C R+1}$

网络函数为:
$$
H(j \omega)=\frac{\dot{U}_{0}}{\dot{U}_{i}}=\frac{1}{\sqrt{1+(\omega R C)^{2}}} \angle-\operatorname{tg}^{-1}(\omega R C)=|H(j \omega)| \angle \varphi(\omega)
$$

式中,$|H(j \omega)| \triangleq\left|\frac{\dot{U}_{o}}{\dot{U}_{i}}\right|=\frac{1}{\sqrt{1+(\omega R C)^{2}}}$,称为幅频特性,显然是低通。 $\varphi(\omega)=-t g^{-1}(\omega R C)$,称为相频特性,显然是输出滞后输入 $t g^{-1}(\omega R C)$ 。

4. 输出超前输入电压的 RC 电路如图 2 所示。 输出电压为:
$$
\dot{U}_{0}=\frac{R}{R+1 / j \omega c} \dot{U}_{i}=\frac{j \omega C R}{j \omega C R+1} \dot{U}_{i}
$$
网络函数为:
$$
H(j \omega)=\frac{\dot{U}_{0}}{\dot{U}_{i}}=\frac{\omega R C}{\sqrt{1+(\omega R C)^{2}}} \angle \frac{\pi}{2}-\operatorname{tg}^{-1}(\omega R C)=|H(j \omega)| \angle \varphi(\omega)
$$
式中,$|H(j \omega)| \triangleq\left|\frac{\dot{U}_{o}}{\dot{U}_{I}}\right|=\frac{\omega R C}{\sqrt{1+(\omega R C)^{2}}}$,称为幅频特性,显然是高通。 $\varphi(\omega)=\frac{\pi}{2}-\operatorname{tg}^{-1}(\omega R C)$,
称为相频特性,显然输出超前输入 $\frac{\pi}{2}-t g^{-1}(\omega R C)$ 。

5. $\mathrm{RC}$ 串并联选频电路
图 3 为 $\mathrm{RC}$ 串并联选频电路。

\section{实验内容及接线图}
1. $\mathrm{RC}$ 低通电路
按图 1 接线, 取 $R=2.2 k \Omega, C=0.1 \mu F, U_{i}=1 V$ (有效值)。测量输出电压,并读取输出电压 $U_{0}=0.707 \mathrm{~V}$ 时的信号频率 $f_{c}$, 用李沙育法测量相位差角,数据记入原始数据部分的表格中。

2. RC 高通电路
按图 2 接线。取 $R=2.2 k \Omega, C=0.1 \mu F, U_{i}=1 V$ (有效值)。测量输出电压 $U_{0}$,并读取 $U_{0}=0.707 V$
时的信号频率 $f_{c}$, 用李沙育法测量相位差角,数据记入原始数据部分的表格中。

3. RC 串并联选频电路
按图 2 接线。取 $R=2.2 k \Omega, C=0.1 \mu F, U_{i}=1 V$ (有效值)。测量输出电压 $U_{0}$, 并读取 $U_{0}=0.707 V$ 时的信号频率 $f_{c}$,用李沙育法测量相位差角,数据记入原始数据部分的表格中。

\section{实验数据及处理}

\section{实验讨论}
1. 测量一组数据时,不要更换测量工具。即同组数据用同种测量仪器测量。

2. 实验测量过程中,对于不同的频率,应保持信号发生器的输出电压恒定不变。

3. 实验中所观察到的椭圆形的李沙育图形,可以通过运算计算出其数学方程,同时证明观察到的图形的正确性。具体计算过程详见 “回答思考题”部分的第二题的证明过程。

\section{思考题}
1. 两个不同频率的正弦量, 能否测量其相位差?为什么?

答: 不同频率的波,测量其相位差是没有意义的。因为其相位差是含时的,即相位差是随时间变化的,不具有稳定值,所以测量相位差是没有意义的。

2. 据你所知,测量频率、振幅和相位差有哪些方法?

答:

测量频率: (1) 用示波器直接测量。(2) 用频率计测量。

测量振幅:(1)用示波器直接测量。(2)用表测量出有效值, 然后计算振幅。

测量相位差:(1) 用示波器, 通过李沙育图形, 计算相位差。(2)

3. 理论证明公式 $\varphi=\sin ^{-1} \frac{B}{A}$ 成立。

答:

设 $C H 1$ 和 $C H 2$ 的正弦波的方程为:
$$
\left\{\begin{array}{l}
x=C \sin (w t+a) \\
y=D \sin (w t+b)
\end{array}\right.
$$
整理如下:
$$
\left(\begin{array}{cc}
\cos a & \sin a \\
\cos b & \sin b
\end{array}\right)\left(\begin{array}{c}
\sin w t \\
\cos w t
\end{array}\right)=\left(\begin{array}{l}
x / C \\
y / D
\end{array}\right)
$$
由线性代数中的 Clamer 法则, 解出:
$$
\left\{\begin{array}{l}
\cos w t=\frac{D x \cos b-C y \cos a}{C D \sin (a-b)} \\
\sin w t=-\frac{D x \sin b-C y \sin a}{C D \sin (a-b)}
\end{array}\right.
$$
由此可得椭圆方程:
$$
(D x \cos b-C y \cos a)^{2}+(D x \sin b-C y \sin a)^{2}=C^{2} D^{2} \sin ^{2}(a-b)
$$
此方程, 表示的是一个非标准的椭圆, 这与实验中观察的结果相吻合。 测量值 $A$ 的两个端点, 是 $C H 1$ 的正弦波的振幅最大的时候的两点:
$$
A_{1}(0,-C), \quad A_{2}(0, C)
$$
由此可计算得 $A$ 值为:
$$
A=2 C
$$
测量值 $B$ 的两个端点, 是 $C H 2$ 的正弦波的振幅最小的时候的两点:
$$
B_{1}(0, C \sin (a-b)), \quad B_{2}(0, C \sin (b-a))
$$
由此可计算得 $B$ 值为:
$$
B=2 C|\sin (a-b)|
$$
由此 $A$ 和 $B$ 值可得:
$$
\begin{gathered}
B / A=\sin (a-b) \\
\Rightarrow \varphi=a-b=\sin ^{-1} \frac{B}{A}
\end{gathered}
$$
证毕

4. 根据实验结果说明选频电路的作用?

答: 有实验结果可知 $R C$ 低通电路的选频作用是低频可被通过, 高频部分不如低频部分易于通过。 $R C$ 高通电路的选频作用是高频可被通过, 低频部分不如高频部分易于通过。 $R C$ 串并联选频电路只能使有限 部分的频段通过, 带宽与所使用的原件有关, 而高频和低频部分不能通过。
\end{document}
