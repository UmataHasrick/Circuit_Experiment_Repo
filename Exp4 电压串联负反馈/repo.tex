\documentclass[lang=cn,11pt]{elegantpaper}

\usepackage{mathrsfs}

\usepackage{slashed}

\title{电压串联负反馈放大器}
\author{}


\institute{中国科学技术大学}

% 不需要版本信息,直接注释即可
%\version{0.07}
% 不需要时间信息的话,需要把 \today 删除。
\date{}


% 如果想修改参考文献样式,请把这行注释掉
\usepackage{gbt7714}  % 国标

\begin{document}

\maketitle

\section{实验目的}
1. 了解反馈放大器的分类和判别方法;

2. 加深理解负反馈对放大器性能的改善作用;

3. 进一步熟悉放大器性能指标的测量方法。

\section{实验原理}
1. 反馈的概念


所谓反馈, 就是指把放大器(或电子系统)输出回路的电信号(电压或电流) 的一部分或全部, 通过一定的方式(反馈网络) 引回到放大器输入回路, 并与输入信号一起参与控制作用的过程。


判䉼放大器是否存在反馈, 要看放大器的输入于输出回路之间有无通路。即反馈网络。它的文件或为输入与输出回路所共有, 或跨接于输入与输出回路之间。 

2.反馈的分类


(1) 按反馈的机型划分:有正反馈和负反馈。


正反缋使放大器的净输入信号 $\dot{X}_{\Sigma}$ 加强, 在信号产生电路中得到广泛应用。


负反馈使放大器的净输入信号 $\dot{X}_{\Sigma}$ 削弱, 但引入负反馈后改善了放大器的性能。但性能的改善都要以降低放大器的増益为代价。


(2) 按反馈网络对输出信号的取样方式划分: 有电压及馈和电流及馈。 


电压反馈是将反缋网络并接在输出端, 即并联取样(电压取样)。反馈信号 $\dot{X}_{f}$ (电压或电流)正比于输出电压 $\dot{V}_{0}$, 电压反馈使得放大器的输出电阻减小, 输出电压得到稳定。


电流反馈是将反馈网络串联在输出端, 即串联取样(电流取样)。反馈信号 $\dot{X}_{f}$ (电压或电流)正比于输出电压 $\dot{I}_{0}$, 电流反馈使放大器的输出电阻増加, 输出电 流得到 稳 定。


(3)) 按及缋信号 $\dot{X}_{f}$ 和输入信号 $\dot{X}_{i}$ 在输入回路的连接方式划分:有串联反馈 和并联反馈。


若输入信号 $\dot{X}_{f}$ 与反馈信号 $\dot{X}_{i}$ 在输入回路中接于不同点, 两者以串联方式影响 净输入信号, 称为串联反馈。


若输入信号 $\dot{X}_{f}$ 与反馈信号 $\dot{X}_{i}$ 在输入回路中接在不同点, 两者以串联方式影响 净输入信号, 称为并联反馈。


由此可见, 采用不同的反馈机型, 可构成正反馈和负反馈放大器。

3. 负及馈放大器的方框图及一般表达式


图 1-10 为负反愦放大器的方框图。图中X表示信号 (电流或者电压), 箭头表示信号传输方向, 表示比较求和。


如图 1-10 可知, 开环放大器的净输入信号为:
$$
\dot{X}_{\Sigma}=\dot{X}_{i}-\dot{X}_{f}
$$
开环放大器的方大倍数(开环増益)为:
$$
\dot{\mathrm{A}}_{0}=\frac{\dot{\mathrm{X}}_{0}}{\dot{\mathrm{X}}_{\Sigma}}
$$
负反馈的反馈系数为:
$$
\mathrm{F}=\frac{\dot{\mathrm{X}}_{\mathrm{f}}}{\dot{\mathrm{X}}_{0}}
$$
反馈放大器的放大倍数(闭环増益)为:
$$
\dot{\mathrm{A}}_{0 \mathrm{f}}=\frac{\dot{\mathrm{X}}_{0}}{\dot{\mathrm{X}}_{\mathrm{i}}}=\frac{\dot{\mathrm{A}}_{0}}{1+\dot{\mathrm{F}} \dot{\mathrm{A}}_{0}}
$$
由上式可知, 加入负反馈后放大器的増益减小了 $\left(1+\dot{\mathrm{F}} \dot{\mathrm{A}}_{0}\right)$ 倍。另 $\mathrm{D}=1+\dot{\mathrm{F}} \dot{\mathrm{A}}_{0}$, 称为反馈深度, 而把 $\dot{\mathrm{F}} \dot{\mathrm{A}}_{0}$ 称为环路増益。 当 $1+\dot{\mathrm{F}} \dot{\mathrm{A}}_{0} \gg 1$ 时,称为深度负反馈。由 此得到:
$$
\dot{\mathrm{A}}_{0 \mathrm{f}}=\frac{\dot{\mathrm{A}}_{0}}{1+\dot{\mathrm{F}} \dot{\mathrm{A}}_{0}} \approx \frac{1}{\dot{\mathrm{F}}}
$$
可见, 在深度负反馈时, 放大倍数仅取决于由反馈网络决定的反馈系数, 几乎与开环放大器无关, 而反馈网络通常由性能稳定的无源元件 $(R, C)$ 组成。因此, 闭环放大器的増益十分的稳定。

4.负反馈对放大器性能的改善

(1)负反馈使放大器的稳定性挺高了D 倍。 

设开环放大器增益的相对变化量为:
$$
\delta=\frac{\Delta A_{0}}{A_{0}}
$$
则闭环放大器增益的相对变化量为:
$$
\delta_{\mathrm{f}}=\frac{\Delta \mathrm{A}_{0 \mathrm{f}}}{\mathrm{A}_{0 \mathrm{f}}}=\frac{\delta}{\mathrm{D}}
$$


(2) 负反馈展宽了放大器的通频带。
设无负反馈时的放大器的上限截至频率和下限截止频率分别为 $\mathrm{f}_{\mathrm{H}}$ 和 $\mathrm{f}_{\mathrm{L}}$, 加入 负反馈后上限截止频率展宽了D 倍:
$$
\mathrm{f}_{\mathrm{Hf}}=\mathrm{D} \cdot \mathrm{f}_{\mathrm{H}}
$$
下限截止频率降低了D 倍:
$$
\mathrm{f}_{\mathrm{Lf}}=\frac{\mathrm{f}_{\mathrm{L}}}{\mathrm{D}}
$$


(3) 负反馈放大器减小了非线性失真, 抑制内部干扰和噪声。 

负反馈只能减小放大器本身造成的非线性失真及内部干扰和噪声。


(4)负反馈可以改善放大器的输入电阻。

设开环放大器的输入电阻为:
$$
\mathrm{R}_{\mathrm{i}}=\frac{\mathrm{V}_{\Sigma}}{\mathrm{I}_{\mathrm{i}}}
$$
加入负载后:
$$
\mathrm{R}_{\mathrm{if}}=\mathrm{DR}_{\mathrm{i}} \cdots \cdots \text { 串联负载反馈 }
$$
$$
\mathrm{R}_{\mathrm{if}}=\frac{\mathrm{R}_{\mathrm{i}}}{\mathrm{D}} \ldots \cdots \text { 并联负载反馈 }
$$


(5) 负反馈可以改善放大器的输入电阻。

设开环放大器的输出电阻为:
$$
\mathrm{R}_{0}=\frac{\mathrm{V}_{0}}{\mathrm{I}_{0}}
$$
加入负载后:
$$
\begin{aligned}
&\mathrm{R}_{0 \mathrm{f}}=\frac{\mathrm{R}_{0}}{\mathrm{D}} \cdots \cdots \text { 电压负载反馈 } \\
&\mathrm{R}_{0 \mathrm{f}}=\mathrm{DR}_{0} \cdots \cdots \text { 电流负载反馈 }
\end{aligned}
$$

\section{思考题}

1,为稳定静态工作点应引入何种反馈?为改善电路动态性能应引入何种反馈?欲增大带负载能力应引入何种反馈?
答:直流负反馈;交流负反馈;电压串联负反馈。


2,反馈网络的负载效应是如何体现在开环放大器中的?
答:引入反馈网络后,对于实验中所使用的电路,相当于引入$R_{F}^{'}$与$R_{f}$并联,
$R_{F}^{'}+R_{f}$与$R_{L}$并联,通过电路中电阻阻值的改变,就相应的改变了放大器的
放大倍数与输出电阻。

\end{document}
