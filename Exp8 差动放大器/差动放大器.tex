%!Mode:: "TeX:UTF-8"
\documentclass[a4paper,11pt,UTF8]{ctexart}

\usepackage{indentfirst} %缩进
\usepackage{xeCJK}    %使用系统字体
\usepackage{fancyhdr} %自定义页眉页脚
\pagestyle{empty}                   %不设置页眉页脚
\usepackage{amsmath, amsthm, amssymb, amsfonts} %数学公式
\usepackage[a4paper,left=3cm,right=3cm,top=3cm,bottom=3cm]{geometry}
%\usepackage[tmargin=1in,bmargin=1in,lmargin=1.25in,rmargin=1.25in]{geometry}.
\usepackage{booktabs} %插入表格
\usepackage[section]{placeins} %避免浮动
\usepackage{listings} %插入代码
\usepackage{ctex}     %中文宏包
\usepackage[svgnames, table]{xcolor} %彩色表格
\usepackage{algorithm}          %伪代码
\usepackage{algorithmicx}
\usepackage{algpseudocode}
\usepackage{algorithm,algpseudocode,float}
\usepackage{lipsum}
\usepackage{enumitem}           %调整列举环境
\usepackage{url}
\usepackage{fontspec,xunicode}
\defaultfontfeatures{Mapping=tex-text} %如果没有它,会有一些 tex 特殊字符无法正常使用,比如连字符。

\usepackage{graphicx}
\graphicspath{{imgs/}}

%%%%%%%%%%%%%%%%%%%%%%%%%%%%%%%%%%%%%%%%%%%%%%%%%%%%%%%%%%%%%%%%
% 缩进及行间距
%%%%%%%%%%%%%%%%%%%%%%%%%%%%%%%%%%%%%%%%%%%%%%%%%%%%%%%%%%%%%%%%
\setlength{\parindent}{22pt} %重新定义缩进长度
\setlength{\baselineskip}{20pt}  %定义行间距
%\renewcommand{\baselinestretch}{1.1} %定义行间距

%%%%%%%%%%%%%%%%%%%%%%%%%%%%%%%%%%%%%%%%%%%%%%%%%%%%%%%%%%%%%%%%
% 列表设置
%%%%%%%%%%%%%%%%%%%%%%%%%%%%%%%%%%%%%%%%%%%%%%%%%%%%%%%%%%%%%%%%
\setenumerate{fullwidth,itemindent=\parindent,listparindent=\parindent,itemsep=0ex,partopsep=0pt,parsep=0ex}
\setenumerate[2]{label=\alph*),leftmargin=1.5em}  %二级item设置
\setitemize{itemindent=38pt,leftmargin=0pt,itemsep=-0.4ex,listparindent=26pt,partopsep=0pt,parsep=0.5ex,topsep=-0.25ex}
\setdescription{itemindent=38pt,leftmargin=0pt,itemsep=-0.4ex,listparindent=26pt,partopsep=0pt,parsep=0.5ex,topsep=-0.25ex}

%%%%%%%%%%%%%%%%%%%%%%%%%%%%%%%%%%%%%%%%%%%%%%%%%%%%%%%%%%%%%%%%
% 图的标题行间距设置
%%%%%%%%%%%%%%%%%%%%%%%%%%%%%%%%%%%%%%%%%%%%%%%%%%%%%%%%%%%%%%%%
\newcommand{\bottomcaption}{%
	\setlength{\abovecaptionskip}{6pt}%
	\setlength{\belowcaptionskip}{6pt}%
	\caption}


%%%%%%%%%%%%%%%%%%%%%%%%%%%%%%%%%%%%%%%%%%%%%%%%%%%%%%%%%%%%%%%%
% 字体定义
%%%%%%%%%%%%%%%%%%%%%%%%%%%%%%%%%%%%%%%%%%%%%%%%%%%%%%%%%%%%%%%%
\setmainfont{Times New Roman}  %默认英文字体.serif是有衬线字体sans serif无衬线字体
\setmonofont{Consolas}
\setCJKmainfont[ItalicFont={楷体}, BoldFont={黑体}]{宋体}%衬线字体 缺省中文字体为
\setCJKsansfont{黑体}
\punctstyle{hangmobanjiao}
%-----------------------xeCJK下设置中文字体------------------------------%
\setCJKfamilyfont{song}{SimSun}                             %宋体 song
\newcommand{\song}{\CJKfamily{song}}
\setCJKfamilyfont{fs}{FangSong}                      %仿宋  fs
\newcommand{\fs}{\CJKfamily{fs}}
\setCJKfamilyfont{ktgb}{KaiTi}                      %楷体2312 ktgb
\newcommand{\ktgb}{\CJKfamily{ktgb}}
\setCJKfamilyfont{yh}{Microsoft YaHei}                    %微软雅黑 yh
\newcommand{\yh}{\CJKfamily{yh}}
\setCJKfamilyfont{hei}{SimHei}                              %黑体  hei
\newcommand{\hei}{\CJKfamily{hei}}
\setCJKfamilyfont{hwxk}{STXingkai}                                %华文行楷  hwxk
\newcommand{\hwxk}{\CJKfamily{hwxk}}
%------------------------------设置字体大小------------------------%
\newcommand{\shiyanbaogao}{\fontsize{36pt}{\baselineskip}\selectfont}
\newcommand{\chuhao}{\fontsize{42pt}{\baselineskip}\selectfont}     %初号
\newcommand{\xiaochuhao}{\fontsize{36pt}{\baselineskip}\selectfont} %小初号
\newcommand{\yihao}{\fontsize{28pt}{\baselineskip}\selectfont}      %一号
\newcommand{\erhao}{\fontsize{21pt}{\baselineskip}\selectfont}      %二号
\newcommand{\xiaoerhao}{\fontsize{18pt}{\baselineskip}\selectfont}  %小二号
\newcommand{\sanhao}{\fontsize{15.75pt}{\baselineskip}\selectfont}  %三号
\newcommand{\sihao}{\fontsize{14pt}{\baselineskip}\selectfont}       %四号
\newcommand{\xiaosihao}{\fontsize{12pt}{\baselineskip}\selectfont}  %小四号
\newcommand{\wuhao}{\fontsize{10.5pt}{\baselineskip}\selectfont}    %五号
\newcommand{\xiaowuhao}{\fontsize{9pt}{\baselineskip}\selectfont}   %小五号
\newcommand{\liuhao}{\fontsize{7.875pt}{\baselineskip}\selectfont}  %六号
\newcommand{\qihao}{\fontsize{5.25pt}{\baselineskip}\selectfont}    %七号

%%%%%%%%%%%%%%%%%%%%%%%%%%%%%%%%%%%%%%%%%%%%%%%%%%%%%%%%%%%%%%%%
% 图题字体大小相同
%%%%%%%%%%%%%%%%%%%%%%%%%%%%%%%%%%%%%%%%%%%%%%%%%%%%%%%%%%%%%%%%
\usepackage{caption}
\captionsetup{font={footnotesize}}   % footnotesize = 9pt
\captionsetup[lstlisting]{font={footnotesize}}

%%%%%%%%%%%%%%%%%%%%%%%%%%%%%%%%%%%%%%%%%%%%%%%%%%%%%%%%%%%%%%%%
% 重定义枚举编号为 1),2)...
%%%%%%%%%%%%%%%%%%%%%%%%%%%%%%%%%%%%%%%%%%%%%%%%%%%%%%%%%%%%%%%%
\renewcommand{\labelenumi}{\theenumi)}

%%%%%%%%%%%%%%%%%%%%%%%%%%%%%%%%%%%%%%%%%%%%%%%%%%%%%%%%%%%%%%%%
% 标题名称中文化
%%%%%%%%%%%%%%%%%%%%%%%%%%%%%%%%%%%%%%%%%%%%%%%%%%%%%%%%%%%%%%%%
\renewcommand\figurename{\hei 图}
\renewcommand\tablename{\hei 表}
\renewcommand\lstlistingname{\hei 代码}
\renewcommand{\algorithmicrequire}{\textbf{输入:}}
\renewcommand{\algorithmicensure}{\textbf{输出:}}
\newtheorem{define}{定义}

%%%%%%%%%%%%%%%%%%%%%%%%%%%%%%%%%%%%%%%%%%%%%%%%%%%%%%%%%%%%%%%%
% 代码设置
%%%%%%%%%%%%%%%%%%%%%%%%%%%%%%%%%%%%%%%%%%%%%%%%%%%%%%%%%%%%%%%%
\lstset{
	columns=fixed,
	numbers=left,                                        % 在左侧显示行号
	numberstyle=\tiny\color{gray},                       % 设定行号格式
	frame=single,                                        % 单线背景边框
	breaklines=true,                                     % 设定LaTeX对过长的代码行进行自动换行
	keywordstyle=\color[RGB]{40,40,255},                 % 设定关键字颜色
	numberstyle=\footnotesize\color{darkgray},
	commentstyle=\it\color[RGB]{0,96,96},                % 设置代码注释的格式
	stringstyle=\rmfamily\slshape\color[RGB]{128,0,0},   % 设置字符串格式
	showstringspaces=false,                              % 不显示字符串中的空格
	language=java,                                        % 设置语言
	basicstyle=\linespread{1.0}\xiaowuhao\ttfamily,                      % 字体字号
	%lineskip=10pt,
	%baselinestretch=1,
}

%%%%%%%%%%%%%%%%%%%%%%%%%%%%%%%%%%%%%%%%%%%%%%%%%%%%%%%%%%%%%%%%
% 伪代码分页
%%%%%%%%%%%%%%%%%%%%%%%%%%%%%%%%%%%%%%%%%%%%%%%%%%%%%%%%%%%%%%%%
\makeatletter
\renewcommand{\ALG@name}{算法}
\newenvironment{breakablealgorithm}
{% \begin{breakablealgorithm}
		\begin{center}
			\refstepcounter{algorithm}% New algorithm
			\hrule height.8pt depth0pt \kern2pt% \@fs@pre for \@fs@ruled
			\renewcommand{\caption}[2][\relax]{% Make a new \caption
				{\raggedright\textbf{\ALG@name~\thealgorithm} ##2\par}%
				\ifx\relax##1\relax % #1 is \relax
				\addcontentsline{loa}{algorithm}{\protect\numberline{\thealgorithm}##2}%
				\else % #1 is not \relax
				\addcontentsline{loa}{algorithm}{\protect\numberline{\thealgorithm}##1}%
				\fi
				\kern2pt\hrule\kern2pt
			}
		}{% \end{breakablealgorithm}
		\kern2pt\hrule\relax% \@fs@post for \@fs@ruled
	\end{center}
}
\makeatother

% =============================================
% Part 1 Edit the info
% =============================================



\newcommand{\course}{电子线路实验(1)}
\newcommand{\newtitle}{差动放大器}

%
\newcommand{\p}{\par}
\newcommand{\np}{\par\noindent}

\title{差动放大器实验}
\author{PB19000132苗立扬  PB18020556 戴佳乐}

% =============================================
% Part 1 Main document
% =============================================
\begin{document}
	
	\maketitle
	\section{实验目的}
	
	\begin{enumerate}
		\item 加深理解差动放大器的工作原理。
		\item 学习差动放大电路主要特性参数的测试方法。
		\item 熟悉基本差动放大电路与具有镜像恒流源差动
		放大电路的性能差别。

	\end{enumerate}
	
	\section{实验原理}
	\subsection{一般放大电路误差的产生}
	在直接耦合式放大电路中,即使将输入端短路,输出也会
	有变化缓慢的输出电压。这种输入电压为零而输出电压
	不为零且缓慢变化的现象称为零点漂移现象。产生该现象的原因主要是晶体三极管的参数受温度的影响,所以零点漂移也
	称为温度漂移,简称温漂。
	
	为了抑制温漂的影响,人们设计出了用两组三极管对称分布对信号进行放大的电路,称为差动放大电路
	\subsection{差动放大电路基本结构}
	典型差动放大电路由两个元件参数相同的基本共射放大电路组成。其基本结构示意图如图\ref{fig:stru}所示。
	\begin{figure}[H]
		\centering
		\fbox{\includegraphics[width=0.5\linewidth]{CCstru.png}}
		\caption{典型差动放大器结构示意图}
		\label{fig:stru}
	\end{figure}
	从示意图中我们可以看到,典型差动放大器有以下结构特征:
	\begin{enumerate}
		\item	电路结构对称
		\item 	有两个输入端,可以单端输入或双端输入信号
		\item	有两个输出端,可以单端输出或双端输出信号
		\item	采用正负电源供电
	\end{enumerate}
	
	差动放大电路抑制温漂原理主要利用两组相同二极管的温漂效应互相抵消。由于电路的对称性, 温度的变化对T1、 T2两管组成的左
	右两个放大电路的影响是一致的, 相当于给两个放大电
	路同时加入了大小和极性完全相同的输入信号。 因此,
	在电路完全对称的情况下, 两管的集电极电位始终相同,
	差动放大电路的输出为零, 不会出现普通直接耦合放大
	电路中的漂移电压, 可见, 差动放大电路利用电路对称
	性抑制了零点漂移现象。
	
	差动放大电路的电路特性不仅表现在对温漂的抑制上。与此同时,电路的对称性也可抑制输入信号的共模成分。我们将分别计算如下。
	\subsection{典型差动放大电路}
	对于典型差动放大电路,我们将射极耦合公共电阻$R_W$来为差动放大电路的两组二极管提供相同的偏置。根据三极管有关知识,我们计算该电路参数如下
	\subsubsection{静态工作点的计算}
	由差动放大电路的结构知,其直流通路左右对称,所以由电压降有关公式以及三极管在放大区工作的放大倍数有关公式$I_C = \beta I_B$,我们计算得到
	\begin{equation}
		\begin{aligned}
		I_B &=\frac{-V_{EE}-V_{BE}}{R_{B1}+2(1+\beta)R_E+(1+\beta)\frac{R_{W1}}{2}}\\
		I_C &= \beta I_B\\
		U_C&=V_{CC}-I_CR_C\\
		U_{CE}&=V_{CC}+|V_{EE}|-I_CR_C-2I_CR_E-I_C\frac{R_{W1}}{2}
		\end{aligned}
	\end{equation}
	
	式中,$I_B$,$I_C$,$U_C$,$U_{CE}$分别为基极电流,集电极电流,集电极电压与集电极与发射极的电压差。$V_{CC}$与$V_{EE}$则分别是已给定的电路直流偏置。

	我们将分别测量给定参数并将他们与理论值相比较。
	
	
	\subsubsection{差动放大电路对差模电压的放大}
	对于差模输入信号,我们将二极管用等效模型替代,得到如图\ref{fig:CDequcir}所示电路图
	\begin{figure}[H]
		\centering
		\fbox{\includegraphics[width=0.5\linewidth]{CDequcir.png}}
		\caption{差模放大电路等效示意图}
		\label{fig:CDequcir}
	\end{figure}
	从该等效电路图中,我们利用三极管有关知识计算得到单端输出电压与双端输出电压增益分别为
	\begin{equation}
		\begin{aligned}
		A_{ud-half}&=\frac{u_{od1}}{u_{id1}}=\frac{u_{od}}{2u_{id}}=\frac{-\beta R_L'}{2[R_{B1}+r_{be}+(1+\beta)\frac{R_{W1}}{2}]}\\
		A_{ud}&=\frac{u_{od}}{u_{id}}=\frac{-\beta R_L'}{R_{B1}+r_{be}+(1+\beta)\frac{R_{W1}}{2}}
		\end{aligned}
	\end{equation}
	
	\subsubsection{差动放大电路对共模电压的抑制}
	对于共模输入信号,我们将二极管用等效模型替代,得到如图\ref{fig:CCequcir}所示电路图
	\begin{figure}[H]
		\centering
		\fbox{\includegraphics[width=0.5\linewidth]{CCequcir.png}}
		\caption{共模放大电路等效示意图}
		\label{fig:CCequcir}
	\end{figure}
	从该等效电路图中,我们利用三极管有关知识计算得到单端输出电压与双端输出电压增益分别为
	\begin{equation}
		\begin{aligned}
		A_{uc-half}&=\frac{u_{oc1}}{u_{ic}}=\frac{-\beta R_L'}{R_{B1}+r_{be}+(1+\beta)\frac{R_{W1}}{2}+2(1+\beta)R_E}\\
		A_{uc}&=\frac{u_{oc1}-u_{oc2}}{u_{ic}}=0
		\end{aligned}
	\end{equation}

	我们引入共模抑制比$K_{CMR} = |\frac{A_{ud}}{A_{uc}}|$来度量典型差动放大电路对共模电压的抑制作用。对于$K_{CMR}$越大的电路,其抑制共模电压的能力越强。
	理论告诉我们,双端输出时共模抑制比可认为等于无穷大
	单端输出时共模抑制比由如下公式给定:
	\begin{equation}
		K_{\mathrm{CMR}} \approx \frac{-\beta R_{C} / 2\left(R_{B 1}+r_{\mathrm{be}}\right)}{-R_{C} / 2 R_{E}} \approx \frac{\beta R_{E}}{R_{B 1}+r_{\mathrm{be}}}
	\end{equation}
	
	\subsection{恒流源差动放大电路}
	 为了更好的抑制输入信号的共模分量,我们将射极耦合公共电阻$R_W$替换为对称的恒流源,电路示意图如图\ref{fig:CCCstru}所示。根据三极管有关知识,我们计算该电路参数。
	 \begin{figure}[H]
	 	\centering
	 	\fbox{\includegraphics[width=0.5\linewidth]{CCCstru.png}}
	 	\caption{共模放大电路等效示意图}
	 	\label{fig:CCCstru}
	 \end{figure}
	
	
	\subsubsection{静态工作点的计算}
	恒流源保证了$I_C$恒定,为
	\begin{equation}
		I_0=-2I_{C1}=-2I_{C2}=-I_R=\frac{V_{EE}+0.7V}{R+R_{W2}+R_{E4}}
	\end{equation}
	计算得到集电极电压为
	\begin{equation}
		U_{C1}=U_{C2}=V_{CC}-I_{C1}R_{C1}=V_{CC}-\frac{I_0R_{C1}}{2}
	\end{equation}
	
	
	\subsubsection{恒流源差动放大电路对差模电压的放大}
	分析类似于典型差动放大电路,得到单端输出电压增益与双端输出电压增益分别为
	
	
	
	\begin{equation}
		\begin{aligned}
			A_{ud-half}&=\frac{u_{od1}}{u_{id1}}=\frac{u_{od}}{2u_{id}}=\frac{-\beta R_L'}{2[R_{B1}+r_{be}+(1+\beta)\frac{R_{W1}}{2}]}\\
			A_{ud}&=\frac{u_{od}}{u_{id}}=\frac{-\beta R_L'}{R_{B1}+r_{be}+(1+\beta)\frac{R_{W1}}{2}}
		\end{aligned}
	\end{equation}
	
	
	\subsubsection{恒流源差动放大电路对共模电压的抑制}
	分析类似于典型差动放大电路,得到单端输出电压增益与双端输出电压增益分别为
	
	\begin{equation}
		\begin{aligned}
			A_{uc-half}&=\frac{u_{oc1}}{u_{ic}}=\frac{-\beta R_L'}{R_{B1}+r_{be}+(1+\beta)\frac{R_{W1}}{2}+2(1+\beta)R_E’}\\
			A_{uc}&=\frac{u_{oc1}-u_{oc2}}{u_{ic}}=0
		\end{aligned}
	\end{equation}
	
	式中$R_E'$为恒流源的交流等效电阻,其大小为$R_{E}^{\prime}=r_{c e 3}\left(1+\frac{\beta R_{\mathrm{E} 3}}{r_{b e 3}+R_{E 3}+R_{B}}\right)$。
	
	\section{实验内容与步骤}
	\subsection{实验内容一}
	测量典型差动放大器的参数,并计算共模抑制比。
	\subsubsection{实验步骤一}
	\begin{figure}[htbp]
		\centering
		\fbox{\includegraphics[width=0.5\linewidth]{nStatic.png}}
		\caption{典型差动放大器静态工作点测量示意图}
		\label{fig:nStatic}
	\end{figure}
	\begin{enumerate}
		\item 按照图 \ref{fig:nStatic}连接线路,将输入信号接地,调节$R_{w1}$电位器使得$V_{C1}=V_{C2}$。用万用表直流电压
		挡位分别测量差动放大器的静态工作点参数。
		\item 将放大器输入端$V_{s2}$接地,从$V_{s1}$输入正弦信号,调节有效值为$V_{id}=20mVrms$,频率$f=1kHz$。测量差模信号的单端及双端输出信号的有效值并用示波器观察波形。
		\item 将输入端$V_{s1}$和$V_{s2}$两点连接在一起,电阻$R_1$与$R_2$从电路中断开,从$V_{s1}$和$V_{s2}$两端输入$V_{ic}=90mVrms$,频率$f=1kHz$的正弦信号。测量共模信号的单端及双端输出信号的有效值并用示波器观察波形。
		\item 根据测量数据计算单端和双端输出的共模抑制比。
	\end{enumerate}
	\subsection{实验内容二}
	将电阻$R_E$改为镜像恒流源,测量构成差动放大器的参数,并计算共模抑制比。
	\subsubsection{实验步骤二}
	\begin{figure}[htbp]
		\centering
		\fbox{\includegraphics[width=0.5\linewidth]{sStatic.png}}
		\caption{恒流源差动放大器静态工作点测量示意图}
		\label{fig:sStatic}
	\end{figure}
	\begin{enumerate}
		\item 按照图 \ref{fig:sStatic}连接线路,将输入信号接地,调节恒流源$R_{w2}=10k$电位器使得$I_0=1mA$即$V_{RC1}=5V$。用万用表直流电压
		挡位分别测量差动放大器的静态工作点参数。
		\item 将放大器输入端$V_{s2}$接地,从$V_{s1}$输入正弦信号,调节有效值为$V_{id}=20mVrms$,频率$f=1kHz$。测量差模信号的单端及双端输出信号的有效值并用示波器观察波形。
		\item 将输入端$V_{s1}$和$V_{s2}$两点连接在一起,电阻$R_1$与$R_2$从电路中断开,从$V_{s1}$和$V_{s2}$两端输入$V_{ic}=90mVrms$,频率$f=1kHz$的正弦信号。测量共模信号的单端及双端输出信号的有效值。
		\item 根据测量数据计算单端和双端输出的共模抑制比。
	\end{enumerate}
	
	\section{实验数据处理与分析}
	\subsection{实验内容1.1}
	我们将两输入口对地短路,调节两输出端输出直流电压差为$1.3mV$,测量得到直流电压实验测得的静态工作点数据如表 \ref{tab:nSTab}所示。可见得到的数据都对称点的值都较为接近,比较理想。
	\begin{table}[!h!tbp]
		\caption{典型差动放大器静态工作点数据}\label{tab:nSTab}
		\centering
		\begin{tabular}{|l|c|c|c|c|c|c|}
			\hline
			数据 &$V_{C_1}$&$V_{C_2}$&$V_{E_1}$&$V_{E_2}$&$V_{B_1}$&$V_{B_2}$         \\ \hline
			值   &$6.3382V$&$6.3543V$&$-0.72606V$&$-0.73339V$&$-27.456mV$&$-32.587mV$     \\ \hline
		\end{tabular}
	\end{table}
	\subsection{误差分析1.1}
	上述静态工作点数据,理论上对称点应完全一致,但由于实验误差,无法做到。但保证了相差小于$20mV$,从原理上可以接受,下面理论计算
	静态工作点数值。计算中采用值为$\beta=150,r_{bb'}=300\Omega,U_{BE}=0.7V$
	\begin{equation}
		\begin{aligned}
			I_{B}&=\frac{-U_{EE}-U_{BE}}{R_{B_1}+2(1+\beta)R_E+(1+\beta)\frac{R_{W_1}}{2}}=3.74\times10^{-3}mA\\
			I_{C}&=\beta I_{B}=0.561mA\\
			U_{CE}&=U_{CC}+U_{EE}-I_CR_C-2I_CR_E-I_C\frac{R_{W_1}}{2}=7.14V\\
			U_C&=U_CC-I_CR_C=6.39V\\
			U_E&=U_C-U_{CE}=-0.75V\\
			U_B&=U_E+U_{BE}=-0.05V\\
			r_{be}&=r_{bb'}+\frac{26mV}{I_B(mA)}=7.25k\Omega
		\end{aligned}
	\end{equation}
	由此可以计算出分别的相对误差(对称点取平均值)为:
	\begin{equation}
		\begin{aligned}
			\left |\frac{\Delta U_C}{U_C}\right |&=6.85‰\\
			\left |\frac{\Delta U_E}{U_E}\right |&=2.70\%\\
			\left |\frac{\Delta U_B}{U_B}\right |&=40.0\%
		\end{aligned}
	\end{equation}
	
	可见,除$U_C$值外,其他静态工作点值基本上只有数量级正确。这主要是因为另外两个点值基本上在$1mV$范围,因此
	毫伏表的测量相当不精确造成的。但$U_C$的误差较小,说明了实验与理论基本上还是对应的。同时理论中采用了大量的近似,并且所选取
	的参数值可能也与实际由偏差,因此这样的误差可以接受。实际上,对称点的电压值难以调整至一致也说明了实验的电压误差基本上是在$20mV$这个数量级的。
	\subsection{实验内容1.2}
	实验中测量得到的数据如表 \ref{tab:ndTab}所示。可见在此精度范围下,单端差模输出的有效值的绝对值基本一致。
	\begin{table}[!h!tbp]
		\caption{典型差动放大器差模输出数据}\label{tab:ndTab}
		\centering
		\begin{tabular}{|l|c|c|c|c|}
			\hline
			数据 &$V_{S_1}$&$V_{od_1}$&$V_{od_2}$&$V_{od}$         \\ \hline
			值   &$20mV$&$618mV$&$608mV$&$1.211V$     \\ \hline
		\end{tabular}
	\end{table}
	由此可以计算出差模信号增益为:
	\begin{equation}
		\begin{aligned}
			\left |A_{ods}\right |=30.90\\
			\left |A_{od}\right |=60.55
		\end{aligned}
	\end{equation}
	\subsection{误差分析1.2}
	可见实际上$v_{od}$的测量值并不完全等于$v_{od1}-V_{od2}$,但实际上这就是毫伏表读数误差造成的。
	为计算理论值,画出单边小信号等效电路如图 \ref{fig:ndSmallSignal}所示。($R_b$与$R_w$未画出,认为负载$R_L\rightarrow\infty$)
	\begin{figure}[htbp]
		\centering
		\fbox{\includegraphics[width=0.5\linewidth]{ndSmallSignal.png}}
		\caption{典型差动放大器差模输出单边小信号模型}
		\label{fig:ndSmallSignal}
	\end{figure}
	则可以求出输出参数为:
	\begin{equation}
		\begin{aligned}
			\left | A_{vd}\right |&=\frac{\beta R_C\parallel\frac{R_L}{2}}{R_{B_1}+r_{be}+(1+\beta)\frac{R_{W_1}}{2}}=70.0\\
			\left | A_{vds}\right |&=\left |\frac{A_{vd}}{2}\right |=35.0
		\end{aligned}
	\end{equation}
	则可求出相对误差为:
	\begin{equation}
		\begin{aligned}
			\left |\frac{\Delta A_{vd}}{A_{vd}}\right |&=13.5\%\\
			\left |\frac{\Delta A_{vds}}{A_{vds}}\right |&=11.7\%\\
		\end{aligned}
	\end{equation}
	首先,可见差模双端输出放大倍数并不是严格的单端输出的两倍。同时测量值与理论值偏差较大,推测是使用计算的$\beta$与$r_{bb'}$不准所导致的。
	这在静态工作点时已有体现,这个误差经过放大变得更加大。同时使用的元件也有一定的误差,小信号分析本身也具有一定误差,并未考虑频率响应。
	\subsection{实验内容1.3}
	实验测得的共模输出数据如表 \ref{tab:ncTab}所示。而输出波形如附图2所示。可见共模输出电压较小。
	
	\begin{table}[!h!tbp]
		\caption{典型差动放大器共模输出数据}\label{tab:ncTab}
		\centering
		\begin{tabular}{|l|c|c|c|c|}
			\hline
			数据 &$U_{i}$&$U_{oc_1}$&$U_{oc_2}$&$U_{oc}$         \\ \hline
			值   &$90mV$&$44.1mV$&$45.3mV$&$12.27mV$     \\ \hline
		\end{tabular}
	\end{table}
	则可以计算出共模电压增益为(单端输出时取平均值):
	\begin{equation}
		\begin{aligned}
			\left | A_{ucs}\right |&=0.4900\\
			\left | A_{uc}\right |&=0.1363
		\end{aligned}
	\end{equation}
	结合上一实验内容,可以算出差模抑制比为:
	\begin{equation}
		\begin{aligned}
			K_{CMRs}&=63.06\\
			K_{CMR}&=444.2
		\end{aligned}
	\end{equation}
	示波器波形附于实验原始数据纸。
	\subsection{误差分析1.3}
	同样有,单端输出和双端输出电压值并不完全匹配的问题。
	单端输出时半边小信号模型如图\ref{fig:ncSmallSignal}所示。($R_b$与$R_w$未画出,认为负载$R_L\rightarrow\infty$)
	\begin{figure}[htbp]
		\centering
		\fbox{\includegraphics[width=0.5\linewidth]{ncSmallSignal.png}}
		\caption{典型差动放大器共模输出单边小信号模型}
		\label{fig:ncSmallSignal}
	\end{figure}
	则可以求出共模输出参数以及结合上一内容模型可以算出差模抑制比为
	\begin{equation}
		\begin{aligned}
			\left | A_{vcs}\right |&\approx\frac{R_c\parallel R_L}{2R_E}=0.500\\
			\left | A_{vc}\right |&=0.00\\
			K_{CMRs}&=\left |\frac{A_{vds}}{A_{vcs}}\right |=70.0\\
			K_{CMR}&=\infty
		\end{aligned}
	\end{equation}
	可见,双端输出时的抑制比并不可能做到理想的正无穷程度,共模输出增益也不可能是0,但前者也非常大,后者也相当小,比较符合理论预测。
	其他量的相对误差为:
	\begin{equation}
		\begin{aligned}
			\left |\frac{\Delta A_{cds}}{A_{cds}}\right |&=2.00\%\\
			\left |\frac{\Delta K_{CMRs}}{K_{CMRs}}\right |&=9.91\%
		\end{aligned}
	\end{equation}
	可见,共模单端输出的相对误差意外得小,这是因为其计算时几乎没有用到三极管的预设参数。侧面证明,计算误差来源可能主要是三极管的参数不准。
	
	单端共模抑制比的误差仍较大,原因如前所述。
	
	此外,我们发现双端输入的$U_{oc}$并不等于我们所设想的0,甚至具有较大的差别。此外,在后一个实验内容中我们对共模电压的测量中,我们的$U_{oc}$也近似为这个值。结合示波器上几乎无法看到的$U_{oc}$的波形,我们有理由相信我们这里通过毫伏表测量得到的结果存在问题。由此,我们分析我们的测量仪器存在较大的噪声影响,带来了测量误差。而这一结论又与我们前面的误差分析遥相呼应,进一步验证了我们的猜想。
	
	\subsection{实验内容2.1}
	实验测得的静态工作点数据如表 \ref{tab:sSTab}所示。可见得到的数据都对称点的值都较为接近,比较理想。
	\begin{table}[!h!tbp]
		\caption{恒流源差动放大器静态工作点数据}\label{tab:sSTab}
		\centering
		\begin{tabular}{|l|c|c|c|c|c|c|c|}
			\hline
			数据 &$V_{C_1}$&$V_{C_2}$&$V_{E_1}$&$V_{E_2}$&$V_{B_1}$&$V_{B_2}$&$R_{W2}$         \\ \hline
			值   &$6.8476V$&$68927V$&$-0.72044V$&$-0.72728V$&$-24.851mV$&$-29.278mV$&$3.8070k\Omega$     \\ \hline
		\end{tabular}
	\end{table}
	\subsection{误差分析2.1}
	下面理论计算静态工作点值,有:
	\begin{equation}
		\begin{aligned}
			I_C=I_0/2=0.500 mA\\
			U_C=U_CC-I_CR_C=7.00V\\
			r_{be}=r_{bb'}+\beta\frac{26mV}{I_C}=8.10k\Omega\\
			U_{B}=-I_bR_B=-0.0227V\\
			U_{E}=U_{B}-U_{BE}=-0.723V
		\end{aligned}
	\end{equation}
	则可以分别计算出相对误差:
	\begin{equation}
		\begin{aligned}
			\left |\frac{\Delta U_C}{U_C}\right |&=1.86\%\\
			\left |\frac{\Delta U_E}{U_E}\right |&=1.18‰\\
			\left |\frac{\Delta U_B}{U_B}\right |&=19.2\%
		\end{aligned}
	\end{equation}
	相对典型差分放大器而言,相对误差有所下降,因为此时理论计算的依赖主要跟$I_0$有关,而其为实验调制出来的,从这里可以猜测误差来源于参数$U_{BE}$的贡献可能最大。
	基本上精度还能接受,$U_C$误差比较小。
	\subsection{实验内容2.2}
	实验中测量得到的数据如表 \ref{tab:sdTab}所示。可见在此精度范围下,单端差模输出的有效值的绝对值基本一致。
	\begin{table}[!h!tbp]
		\caption{恒流源差动放大器差模输出数据}\label{tab:sdTab}
		\centering
		\begin{tabular}{|l|c|c|c|c|}
			\hline
			数据 &$V_{S_1}$&$V_{od_1}$&$V_{od_2}$&$V_{od}$         \\ \hline
			值   &$20mV$&$594mV$&$594mV$&$1.178V$     \\ \hline
		\end{tabular}
	\end{table}
	可见对称度较典型的低,但$V_{od}$的值与单端输出的值匹配较好。
	\begin{equation}
		\begin{aligned}
			\left |A_{ods}\right |=29.7\\
			\left |A_{od}\right |=58.9
		\end{aligned}
	\end{equation}
	示波器波形附于实验原始数据纸。
	\subsection{误差分析2.2}
	理论分析与分析1.2相似,小信号模型也类似,故不加赘述。
	\begin{equation}
		\begin{aligned}
			\left | A_{vd}\right |&=\frac{\beta R_C\parallel\frac{R_L}{2}}{R_{B_1}+r_{be}+(1+\beta)\frac{R_{W_1}}{2}}=66.8\\
			\left | A_{vds}\right |&=\left |\frac{A_{vd}}{2}\right |=33.4
		\end{aligned}
	\end{equation}
	则可求出相对误差为:
	\begin{equation}
		\begin{aligned}
			\left |\frac{\Delta A_{vd}}{A_{vd}}\right |&=11.8\%\\
			\left |\frac{\Delta A_{vds}}{A_{vds}}\right |&=11.1\%\\
		\end{aligned}
	\end{equation}
	问题与典型时一致,应该与三极管参数以及毫伏表的测量误差相关。
	\subsection{实验内容2.3}
	实验测得的共模输出数据如表 \ref{tab:ncTab}所示。
	
	\begin{table}[!h!tbp]
		\caption{典型差动放大器共模输出数据}\label{tab:ncTab}
		\centering
		\begin{tabular}{|l|c|c|c|c|}
			\hline
			数据 &$U_{i}$&$U_{oc_1}$&$U_{oc_2}$&$U_{oc}$         \\ \hline
			值   &$90mV$&$5.69mV$&$6.49mV$&$11.62mV$     \\ \hline
		\end{tabular}
	\end{table}
	则可以计算出共模电压增益为(单端输出时取平均值):
	\begin{equation}
		\begin{aligned}
			\left | A_{ucs}\right |&=0.06322\\
			\left | A_{uc}\right |&=0.1291
		\end{aligned}
	\end{equation}
	结合上一实验内容,可以算出差模抑制比为:
	\begin{equation}
		\begin{aligned}
			K_{CMRs}&=469.8\\
			K_{CMR}&=456.2
		\end{aligned}
	\end{equation}
	\subsection{误差分析2.3}
	为计算共模输出,需要先求出恒流源等效电阻。但由于三极管参数未知,取$r_{ce}=100k\Omega$,则有
	\begin{equation}
		\begin{aligned}
			R_B&\approx(R+R_{W2})\parallel R_{E4}=1.64k\Omega\\
			r_{be3}&=r_{bb'}+(1+\beta)\frac{26mV}{I_{E3}mA}=4.23k\Omega\\
			R_e'&=r_{ce}(1+\frac{\beta R_{E_3}}{r_{be3}+R_{E3}+R_{B}})=3.91M\Omega
		\end{aligned}
	\end{equation}
	则由此计算出的共模增益以及抑制比为
	\begin{equation}
		\begin{aligned}
			\left | A_{vcs}\right |&\approx\frac{R_c\parallel R_L}{2R_e'}\approx1\times10^{-3}\\
			\left | A_{vc}\right |&=0.00\\
			K_{CMRs}&=\left |\frac{A_{vds}}{A_{vcs}}\right |=3.34\times10^{4}\\
			K_{CMR}&=\infty
		\end{aligned}
	\end{equation}
	可见,基本上单端输出时数量级上差了一个数量级,故不再计算相对误差。这是因为此时的测量mV级已经相当不准确了。
	此处我们的$V_{oc}$数值与实验一对应部分相似,同样的还有示波器的无法观测到的波形,进一步说明了实验噪声与毫伏表测量误差的影响。
	\section{实验总结}
	本次实验测量了差动放大器的典型形态和电流源形态。可见使用恒流源作为负载之后,共模抑制比大幅度上升,这是因为恒流源等效输出电阻是非常大的。从理论分析上讲,
	与实验相比,精度比较一般。但在当前实验条件以及测量精度下,可以认为还是符合条件的,因此基本上的结果还是符合预期。
	\section{实验思考题}
	\subsection{为什么要对差分放大器进行调零,在实验中是否非常重要?}
	\np \textbf{答:}
	\p 理论上电路的左右两端需要保持完全一致,才能保证在零输入时输出为 0,但实际上,因为左右元件的参数不能做到完全一致,所以 需要加入一个可调的元件来修正这种误差。
	\p 如果未进行调零或者调零调的不精确,那么在输入共模信号时,元件不对称带来的误差会使得 $u_{c1}$ 与 $u_{c2}$ 误差变大。这会使得 $u_c$ 变大。因为共模输出电压本来应该很小(1mV量级),如果这个值变大了,那么 $K_{CMR}=\frac{|u_d|}{|u_c|}$ 会变大非常多。这也意味着差动放大器的共模抑制比急剧下降。本来差分放大器就是用来抑制共模影响信号的(如零漂、温漂),这些误差值本来也不是太大。假如因为差动放大器未调零而使得输出共模信号很大,那么使用差动放大电路就没有任何优势了。
	
	\subsection{差分放大器的差模输入电压是与输入电压的差还是与输入电压成正比?}
	\np \textbf{答:}
	\p 假设输出是双端的,但值不相同,为 $u_{i1}$ 和 $u_{i2}$(单端输入是包括在内的,只需要令其中一个电压为 0)。
	\p 那么共模信号为 $u_c=\frac{u_{i1}+u_{i2}}{2}$,差模信号为 $u_d=\frac{u_{i1}-u_{i2}}{2}$,输入信号是二者叠加。而输出信号中共模成分基本被抑制,仅剩下差模成分,假设单端输出差模放大系数为 $A_{ds}$,那么输出电压为 $u_o=A_{ds}\cdot 2u_d=A_{ds}(u_{i1}-u_{i2})$。所以显然输出电压是与输入电压的差值成正比的。只放大差模信号,这也正是使用差动放大器的优势所在。
	
	
	\subsection{典型差动放大电路与恒流源差动放大电路在观测 $u_{c1}$ 与 $u_{c2}$ 的波形时,其大小、极性及共模抑制比 $K_{CMR}$ 有何区别?为什么?}
	\np 答:
	\p 在共模时,只需要考虑一半的电路,恒流源等效于一个大电阻 $R$。如图所示:
	\begin{figure}[H]
		\centering
		\includegraphics[width=3.5cm]{CCircuit}
		\caption{左边无恒流源,右边有恒流源}
		\label{fig:add-3}
	\end{figure}
	\p 其交流小信号模型为:
	\begin{figure}[H]
		\centering
		\includegraphics[width=10cm]{CCircuitSmallSignal}
		\caption{左边无恒流源,右边有恒流源}
		\label{fig:add-3-ss}
	\end{figure}
	\p 容易计算出放大系数:
	\begin{itemize}
		\item 无恒流源:
		\np 
		\[ A_{vc}=\frac{-(\beta+1)R_c}{r_{be}+(1+\beta)\cdot 2r_o} \]
		\item 有恒流源:
		\np
		\[ A_{vc}=\frac{-(\beta+1)R_c}{r_{be}+(1+\beta)\cdot(2r_o+R)} \]
	\end{itemize}
	\p 因为 $R$ 是一很大的电阻,所以有恒流源时 $A_{vc}$ 比无恒流源时小得多,因此起到了进一步抑制共模信号的作用,可以进一步提高 $K_{CMR}$。
	\p 综上,根据理论,加入恒流源后,共模输出信号显著减小,极性不变,共模抑制比显著提高。
	\p 实际实验时,通过加入恒流源,确实能把 $u_{co}$ 从约 1mV 降到 0(小于毫伏表量程,也无法用示波器看见)。$K_{CMR}$ 急剧提高,直至近乎 $+\infty$。示波器上图像是 0 附近模糊的一片,并没有清晰的图像(参见 \ref{fig:add-3-real}),据估计,可能是因为输出信号太小,噪声等干扰信号占据了主导地位。所以,无法验证两种输出信号是否极性一致。
	
	
	
\end{document}